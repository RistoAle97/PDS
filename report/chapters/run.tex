\section{Running the project}
To build everything needed to test the project you can make use of the Makefile just by typing \textit{make} once you have extracted the zip file.
\vspace{3mm}

Afterwards, you can generate datasets of points with \textit{generate\_points.py} using it in the following way (it will create 10000 points by using a uniform distribution between 0 and 10 if no arguments are passed):
\begin{minted}[fontsize=\small, bgcolor=gray!30]{bash}
python generate_points.py --n 100000 --file datasets/points.txt
\end{minted}

Then you can run every implementation easily, the following are just examples:
\begin{minted}[fontsize=\small, bgcolor=gray!30]{bash}
./knn_sequential datasets/points.txt 10
./knn_parallel datasets/points.txt 10 256
./knn_ff datasets/points.txt 10 256
\end{minted}

For each implementation the second argument is the value of k, meanwhile the third argument, for both stdlib threads and FastFlow implementations, is the number of workers.
If you want to execute more times the same implementation or to execute more than one at time, use the \textit{compute\_knn.py} method:
\begin{minted}[fontsize=\small, bgcolor=gray!30]{bash}
python --file datasets/points_10k.txt --k 10 --nw 256 --runs 10 --execute spf
\end{minted}
The last argument specifies what versions to run (s for sequential, p for parallel and f for FastFlow).